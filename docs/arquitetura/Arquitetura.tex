\chapter{Arquitetura}
\begin{figure}[H]
	\centering
		\includegraphics[width=\linewidth]{arquitetura}
	\caption{Em azul estão as interfaces e classes que devem ser 
	implementadas por um programador que esteja interessado em implementar 
um algoritmo no módulo comum. As outras cores representam interfaces e classes 
que devem ser implementadas para criar-se um novo módulo sobre o módulo comum.}
	\label{fig:arquitetura}
\end{figure}

Para que se torne possível a utilização das interfaces disponibilizadas pelo 
módulo comum existe a necessidade de realizar um conjunto de configurações. 
Devido à necessidade de se realizar as configurações e para que essas 
configurações sejam feitas de forma independente da plataforma, 
criou-se um \texttt{CommonConfig} que tem a responsabilidade de registar as 
implementações comuns num \texttt{ModuleConfig}. Os módulos respetivos às 
plataformas, que estão a mapear para o módulo comum, têm de no 
\texttt{ModuleConfig} proceder às configurações necessárias para a respetiva 
plataforma. Esta implementação implica para que se use o módulo comum se proceda 
ao uso do CommonConfig.

O módulo comum foi construído de modo a que se possa fazer proveito das 
interfaces programáveis disponibilizas pelas plataformas Apache Giraph e Apache 
Hama. Usando o módulo comum é possível usar tipos de mensagens e tipos de valor 
para vértices diferentes, mesmo usando a plataforma Apache Hama onde isto não é 
possível.

Contudo, nem todas as funcionalidades são disponibilizadas pelo módulo comum. 
Uma das funcionalidades que não foi conseguida até ao momento foi mapear os 
agregadores devido à diferença dos detalhes da sua implementação entre as duas 
plataformas. Apesar de não ser possível a implementação de agregadores comuns às 
duas plataformas, é possível utilizar agregadores cujo tipo do valor a agregar é 
diferente do tipo do valor do vértice, o que não era possível no Hama. Existe 
ainda outra funcionalidade que não foi mapeada, o \texttt{MasterCompute}, devido 
a não existir implementação no Hama. O \textit{input/output} é diferente nas 
duas plataformas daí que não exista nenhuma abstração no módulo comum.