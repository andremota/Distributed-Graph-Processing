\section{\textit{Betweenness Centrality}}

O algoritmo de \textit{Betweeness Centrality} é muito utilizado para o estudo 
de 
redes sociais devido a indicar a medida de centralidade para cada vértice da 
rede a que foi aplicado,isto é, calcula para cada vértice um grau de 
importância/influência. 

Para se calcular a centralidade de cada vértice é necessário calcular os 
caminhos mais curtos de todos para todos. A formula para calcular a 
\textit{Betweeness Centrality} para um dado vértice v é a seguinte:
\begin{center}
	\begin{equation}
		Bc(v) = \sum\limits_{s \neq v \neq t} 
\frac{\sigma_{st}(v)}{\sigma_{st}}
		\label{eq:bc}
	\end{equation}
	$\sigma_{st}-$~caminhos mais curtos do vértice s para o vértice t.\\
	$\sigma_{st}(v)-$ $\sigma_{st}$ dos quais passam em v.\\
\end{center}

\subsection{Algoritmo para calcular a Betweenness Centrality}
Para calcular a \textit{Betweenness Centrality} pode-se usar um algoritmo de 
\textit{Breath First Search} (BFS) modificado de modo a calcular os caminhos 
mais curtos de todos para todos. 

\subsection{Algoritmo distribuído para calcular a Betweenness Centrality}
O algoritmo distribuído consiste em calcular os caminhos mais curtos de todos 
para todos inicialmente de forma paralela, sendo parecido à implementação do 
\textit{Shortest Path} distribuído. Após se obter todos os caminhos mais curtos 
então pode-se proceder ao cálculo da \textit{Betweeness Centrality}. 

Para o algoritmo distribuído é necessário a existência de dois tipos de 
mensagens. Um dos tipos de mensagem tem de ter informação sobre o vértice de 
começo, o vértice que enviou a mensagem e a distância. Tem também de existir 
uma mensagem que indica que os vértices fazem parte de um caminho mais curto. 
Cada vértice terá informação acerca dos seus predecessores e a distância 
mínima.O algoritmo distribuído pode ser descrito da seguinte forma:
\begin{algorithm}
    \begin{minipage}{\linewidth}  \begin{enumerate}
      \item Começa-se por enviar para os vértices adjacentes a 
mensagem de começo com distância 0.
      \item Iterar sobre as mensagens recebidas e verificar se são ou não 
caminhos mais curtos\footnote{Tendo em conta que não se está a considerar 
pesos, a primeira mensagem recebida (proveniente de um vértice de começo) é 
considerada o caminho mais curto.}. Caso seja um caminho mais curto então 
adicionar à lista de predecessores do vértice de começo o vértice que enviou a 
mensagem (caso o vértice de começo for diferente do vértice que enviou a 
mensagem) e . 
      \item 
    \end{enumerate}
  \end{minipage}
\end{algorithm} de

