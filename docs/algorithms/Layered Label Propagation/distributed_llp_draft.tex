\documentclass[a4paper,10pt]{report}
\usepackage[utf8]{inputenc}
\usepackage{graphicx}

% Title Page
\title{}
\author{}


\begin{document}

\section*{\textit{Layered Label Propagation Distribúido}(Esboço)}

Cada vértice terá:
\begin{itemize}
  \item A sua \textit{label}
  \item $v_i$ associado da sua $label_i$.
\end{itemize}

\begin{figure}[h]
\center
\includegraphics{graph_step0}
\caption{Grafo de exemplo.\label{fig:distributedexample1}}
\end{figure}

{\bf Exemplo para o grafo da figura \ref{fig:distributedexample1}}
\\[0.25cm]
{\bf Fase de preparação:}\\
Todos os vértices ficam com uma $label$ única. 
\\[0.25cm]
{\bf 1º \textit{Superstep} (1º Passo do algoritmo):}
Não receberam mensagens, logo não precisam de fazer \textit{update} ao sei 
$v_i$.
Os vértices enviam para os seus adjacentes 
informação sobre a sua \textit{label} e com $v_i=1$.
\\[0.25cm]
{\bf 2º \textit{Superstep} (2º Passo do algoritmo):}
Os vértices recebem as mensagens enviadas pelos seus adjacentes e calculam a 
$label_i$ que é maximizada na sua vizinhança.
\\[0.25cm]
Vértice A:
  \begin{tabular}{| l | l | l | l |}
  \hline
  $label_i$ & $k_i$ & $v_i$ & $k_i - \gamma(v_i - k_i)$\\ \hline
  A & 0 & 1 & -1 \\ \hline
  B & 1 & 1 & 1  \\ \hline
  C & 1 & 1 & 1  \\ \hline
  D & 1 & 1 & 1  \\ \hline
  \end{tabular}
  A fica com a \textit{label} B.
\\[0.25cm]
Vértice B:
  \begin{tabular}{| l | l | l | l | l |}
  \hline
  $label_i$ & $k_i$ & $v_i$ & $k_i - \gamma(v_i - k_i)$\\ \hline
  A & 1 & 1 & 1  \\ \hline
  B & 0 & 1 & -1 \\ \hline
  C & 1 & 1 & 1  \\ \hline
  D & 1 & 1 & 1  \\ \hline
  \end{tabular}
  B fica com a \textit{label} A.
\\[0.25cm]
Vértice C:
  \begin{tabular}{| l | l | l | l |}
  \hline
  $label_i$ & $k_i$ & $v_i$ & $k_i - \gamma(v_i - k_i)$\\ \hline
  B & 1 & 1 & 1 \\ \hline
  C & 0 & 1 & -1 \\ \hline
  D & 1 & 1 & 1 \\ \hline
  \end{tabular}
  C fica com a \textit{label} A.
\\[0.25cm]
Vértice D:
  \begin{tabular}{| l | l | l | l |}
  \hline
  $label_i$ & $k_i$ & $v_i$ & $k_i - \gamma(v_i - k_i)$\\ \hline
  A & 1 & 1 & 1  \\ \hline
  B & 1 & 1 & 1  \\ \hline
  C & 1 & 1 & 1  \\ \hline
  D & 0 & 1 & -1 \\ \hline
  \end{tabular}  
  D fica com a \textit{label} A.
\\[0.25cm]
Vértice E:
  \begin{tabular}{| l | l | l | l |}
  \hline
  $label_i$ & $k_i$ & $v_i$ & $k_i - \gamma(v_i - k_i)$\\ \hline
  E & 0 & 1 & -1 \\ \hline
  B & 1 & 1 & 1  \\ \hline
  F & 1 & 1 & 1  \\ \hline
  J & 1 & 1 & 1  \\ \hline
  \end{tabular}  
  E fica com a \textit{label} B.
\\[0.25cm]
Vértice F:
  \begin{tabular}{| l | l | l | l |}
  \hline
  $label_i$ & $k_i$ & $v_i$ & $k_i - \gamma(v_i - k_i)$\\ \hline
  F & 0 & 1 & -1 \\ \hline
  E & 1 & 1 & 1  \\ \hline
  I & 1 & 1 & 1  \\ \hline
  G & 1 & 1 & 1  \\ \hline
  H & 1 & 1 & 1  \\ \hline
  \end{tabular}  
  F fica com a \textit{label} E.
\\[0.25cm]
Vértice G:
  \begin{tabular}{| l | l | l | l |}
  \hline
  $label_i$ & $k_i$ & $v_i$ & $k_i - \gamma(v_i - k_i)$\\ \hline
  G & 0 & 1 & -1  \\ \hline
  I & 1 & 1 & 1 \\ \hline
  F & 1 & 1 & 1  \\ \hline
  H & 1 & 1 & 1  \\ \hline
  \end{tabular}  
  G fica com a \textit{label} F.
\\[0.25cm]
Vértice H:
  \begin{tabular}{| l | l | l | l |}
  \hline
  $label_i$ & $k_i$ & $v_i$ & $k_i - \gamma(v_i - k_i)$\\ \hline
  H & 1 & 1 & -1  \\ \hline
  I & 0 & 1 & 1 \\ \hline
  F & 1 & 1 & 1  \\ \hline
  G & 1 & 1 & 1  \\ \hline
  \end{tabular}  
  H fica com a \textit{label} F.
\\[0.25cm]
Vértice I:
  \begin{tabular}{| l | l | l | l |}
  \hline
  $label_i$ & $k_i$ & $v_i$ & $k_i - \gamma(v_i - k_i)$\\ \hline
  I & 0 & 1 & -1 \\ \hline
  F & 1 & 1 & 1  \\ \hline
  G & 1 & 1 & 1  \\ \hline
  H & 1 & 1 & 1  \\ \hline
  \end{tabular}  
  I fica com a \textit{label} F.
\\[0.25cm]
Vértice J:
  \begin{tabular}{| l | l | l | l |}
  \hline
  $label_i$ & $k_i$ & $v_i$ & $k_i - \gamma(v_i - k_i)$\\ \hline
  J & 0 & 1 & -1 \\ \hline
  E & 1 & 1 & 1  \\ \hline
  K & 1 & 1 & 1  \\ \hline
  L & 1 & 1 & 1  \\ \hline
  M & 1 & 1 & 1  \\ \hline
  N & 1 & 1 & 1  \\ \hline
  \end{tabular}  
  J fica com a \textit{label} E.
\\[0.25cm]
Vértice K:
  \begin{tabular}{| l | l | l | l |}
  \hline
  $label_i$ & $k_i$ & $v_i$ & $k_i - \gamma(v_i - k_i)$\\ \hline
  K & 0 & 1 & -1 \\ \hline
  J & 1 & 1 & 1  \\ \hline
  L & 1 & 1 & 1  \\ \hline
  M & 1 & 1 & 1  \\ \hline
  N & 1 & 1 & 1  \\ \hline
  \end{tabular}  
  K fica com a \textit{label} J.
\\[0.25cm]
Vértice L:
  \begin{tabular}{| l | l | l | l |}
  \hline
  $label_i$ & $k_i$ & $v_i$ & $k_i - \gamma(v_i - k_i)$\\ \hline
  L & 0 & 1 & -1 \\ \hline
  J & 1 & 1 & 1  \\ \hline
  K & 1 & 1 & 1  \\ \hline
  M & 1 & 1 & 1  \\ \hline
  N & 1 & 1 & 1  \\ \hline
  \end{tabular}  
  L fica com a \textit{label} J.
\\[0.25cm]
Vértice M:
  \begin{tabular}{| l | l | l | l |}
  \hline
  $label_i$ & $k_i$ & $v_i$ & $k_i - \gamma(v_i - k_i)$\\ \hline
  M & 0 & 1 & -1 \\ \hline
  J & 1 & 1 & 1  \\ \hline
  K & 1 & 1 & 1  \\ \hline
  L & 1 & 1 & 1  \\ \hline
  N & 1 & 1 & 1  \\ \hline
  \end{tabular}  
  M fica com a \textit{label} J.
\\[0.25cm]
Vértice N:
  \begin{tabular}{| l | l | l | l |}
  \hline
  $label_i$ & $k_i$ & $v_i$ & $k_i - \gamma(v_i - k_i)$\\ \hline
  N & 0 & 1 & -1 \\ \hline
  J & 1 & 1 & 1  \\ \hline
  K & 1 & 1 & 1  \\ \hline
  L & 1 & 1 & 1  \\ \hline
  M & 1 & 1 & 1  \\ \hline
  \end{tabular}  
  N fica com a \textit{label} J.
\\[0.25cm]
{\bf 3º \textit{Superstep} (3º Passo do algoritmo):} Os vértices representantes 
de cada comunidade calculam o $v_i$ atualizado para a sua $label_i$ e devolvem 
para os seus membros.
\\[0.25cm]
Vértice A:
  \begin{tabular}{| l | l | l | l |}
  \hline
  $From Vid$ & $label_i$ & $v_i$ \\ \hline
  B & A & 2 \\ \hline
  C & A & 2 \\ \hline
  D & A & 2 \\ \hline
  \end{tabular} 
\\[0.25cm]
  Recebeu uma mensagem de B, e B é a sua comunidade atual, logo escolhe a menor 
$label$. A fica com a \textit{label} A.
  Envia-se $v_A=4$ para os vértices: A,B,C,D.
\\[0.25cm]
Vértice B:
  \begin{tabular}{| l | l | l | l |}
  \hline
  $From Vid$ & $label_i$ & $v_i$ \\ \hline
  A & B & 2 \\ \hline
  E & B & 2 \\ \hline
  \end{tabular}  
\\[0.25cm]
  B recebeu uma mensagem de A, sendo A comunidade de B, daí ignora-se a mensagem 
proveniente de A. Porque a $label$ do vértice A é maior do que a $label$ 
de B.
  Envia-se $v_B = 1$ para o vértice E.
\\[0.25cm]
Vértice E:
  \begin{tabular}{| l | l | l | l |}
  \hline
  $From Vid$ & $label_i$ & $v_i$ \\ \hline
  F & E & 2 \\ \hline
  J & E & 2 \\ \hline
  \end{tabular}  
\\[0.25cm]
  B recebeu uma mensagem de A, sendo A comunidade de B, daí ignora-se a 
mensagem 
proveniente de A. Porque a $label$ do vértice A é maior do que a $label$ 
de B.
  Envia-se $v_E = 1$ para os vértices: F,J.
\\[0.25cm]
Vértice F:
  \begin{tabular}{| l | l | l | l |}
  \hline
  $From Vid$ & $label_i$ & $v_i$ \\ \hline
  I & F & 2 \\ \hline
  G & F & 2 \\ \hline
  H & F & 2 \\ \hline
  \end{tabular}  
\\[0.25cm]
  Envia-se $v_F = 3$ para os vértices: I, G, H.
\\[0.25cm]
Vértice J:
  \begin{tabular}{| l | l | l | l |}
  \hline
  $From Vid$ & $label_i$ & $v_i$ \\ \hline
  K & J & 2 \\ \hline
  L & J & 2 \\ \hline
  M & J & 2 \\ \hline
  N & J & 2 \\ \hline
  \end{tabular}  
\\[0.25cm]
  Envia-se $v_J = 4$ para os vértices: K, L, M,N.
\\[0.25cm]
{\bf 4º \textit{Superstep} (1º Passo do algoritmo):}
Receberam mensagens, logo iram meter o seu $v_i$ igual ao que receberam.
Os vértices enviam para os seus adjacentes 
informação sobre a sua \textit{label} e com $v_i=1$.

\begin{tabular}{| l | l | l |}
    \hline
    $Vid$ & $label_i$ & $v_i$ \\ \hline
    A & A & 4 \\ \hline
    B & A & 4 \\ \hline
    C & A & 4 \\ \hline
    D & A & 4 \\ \hline
    E & B & 1 \\ \hline
    F & E & 2 \\ \hline
    G & F & 3 \\ \hline
    H & F & 3 \\ \hline
    I & F & 3 \\ \hline
    J & E & 2 \\ \hline
    K & J & 4 \\ \hline
    L & J & 4 \\ \hline
    M & J & 4 \\ \hline
    N & J & 4 \\ \hline
\end{tabular}
\\[0.25cm] 
{\bf 5º \textit{Superstep} (2º Passo do algoritmo):}
Volta-se a calcular a $label$ com que o vértice irá ficar.

Repete-se o processo até que nenhum vértice mude no 2º Passo do algoritmo.

\end{document}          
