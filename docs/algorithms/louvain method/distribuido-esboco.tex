\documentclass[a4paper,10pt]{report}
\usepackage[utf8]{inputenc}
\usepackage{mathtools}

\begin{document}

\section*{Esboço}

Cada vértice tem:
\begin{itemize}
	\item $K_i$
	\item Histórico(id,$\sum_{tot}$) das suas comunidades
	\item Histórico das comunidades dos vizinhos
\end{itemize}
E para cada vizinho irá calcular o $K_{i,in}$ e o ganho em relação à comunidade do vizinho.


Então considerando o grafo na figura~\ref{fig:esb1}.

\begin{figure}
\includegraphics[width=20mm]{esboco1}
\caption{Primeiro grafo\label{fig:esb1}}
\end{figure}


\begin{table}[h]
	\centering
		\begin{tabular}{ l | l | l | l }
		\hline
			Vértice & Ki & Histórico Comunidades & Histórico Vizinhos \\ \hline
			a & 2 & (a,2) & - \\ \hline
			b & 2 & (b,2) & - \\ \hline
			c & 2 & (c,2) & - \\ \hline
			d & 2 & (d,2) & - \\ \hline
		\end{tabular}
\end{table}

Como o ganho (0.5) é igual para todos como regra de desempate os vértices escolhem a comunidade com o id menor, ficando cada vértice com os seguintes valores:

\begin{table}[h]
	\centering
		\begin{tabular}{ l | l | l | l }
		\hline
			Vértice & Ki & Histórico Comunidades & Histórico Vizinhos \\ \hline
			a & 2 & (a,2), (a,4) &  b: (b,2), d: (d,2)\\ \hline
			b & 2 & (b,2), (a,4) &  a: (a,2), c: (c,2)\\ \hline
			c & 2 & (c,2), (b,4) &  b: (b,2), d: (d,2)\\ \hline
			d & 2 & (d,2), (a,4) &  a: (a,2), c: (c,2)\\ \hline
		\end{tabular}
\end{table}

O que leva as três comunidades referidas na figura~\ref{fig:esb2}.

\begin{figure}
\includegraphics[width=20mm]{esboco2}
\caption{Primeiro grafo com as comunidades\label{fig:esb2}}
\end{figure}

Se no próximo passo cada vértice enviar para os seus vizinhos a sua comunidade atual e a comunidade dos seus outros vizinhos e ao receber esta informação decidir que, como o seu vizinho decidiu juntar-se a outra comunidade, irá também juntar-se a comunidade do vizinho atualizando o seu $\sum_{tot}$ de acordo. Faz isso para todos os vizinhos, ou seja, é feito uma união entre todas as comunidades.


No exemplo apresentado anteriormente basta enviar o histórico local dos vizinhos para garantir que o $\sum_{tot}$ da comunidade está atualizado em todos os vértices da comunidade. Podendo começar a enviar esta nova informação aos vizinhos não pertencentes à mesma comunidade e recomeçar.

\begin{figure}
\includegraphics[width=20mm]{esboco3}
\caption{Segundo grafo\label{fig:esb3}}
\end{figure}

\newpage
 Mas referindo agora ao grafo na figura~\ref{fig:esb3}. Tal como no primeiro grafo o ganho (2/3) é igual para todos os vértices. E depois do primeiro cálculo cada vértice fica com os seguintes valores:

\begin{table}[h]
	\centering
		\begin{tabular}{ l | l | l | l }
		\hline
			Vértice & Ki & Histórico Comunidades & Histórico Vizinhos \\ \hline
			a & 2 & (a,2), (a,4) &  b: (b,2), f: (f,2)\\ \hline
			b & 2 & (b,2), (a,4) &  a: (a,2), c: (c,2)\\ \hline
			c & 2 & (c,2), (b,4) &  b: (b,2), d: (d,2)\\ \hline
			d & 2 & (d,2), (c,4) &  c: (c,2), e: (e,2)\\ \hline
			e & 2 & (e,2), (d,4) &  d: (d,2), f: (f,2)\\ \hline
			f & 2 & (f,2), (a,4) &  a: (a,2), e: (e,2)\\ \hline
		\end{tabular}
\end{table}


O que leva as comunidades representadas na figura~\ref{fig:esb4}. Mas ao contrário do grafo anterior não é possível fazer uma união de todas as comunidades numa só porque existem vértices que nunca recebem informação sobre outros. Por exemplo: O vértice \texttt{a} não sabe que o vértice \texttt{d} existe porque embora \texttt{c} envie essa informação a \texttt{b}, \texttt{b} só poderia enviar esta nova informação no próximo passo. Mas não há como saber se deve ou não continuar a enviar informação para ser realizada uma união das comunidades ou se deve começar a calcular o ganho outra vez.

Uma maneira possível para resolver este problema seria que a fase de união de comunidades só acaba quando os vértices vizinhos ou fazem parte da mesma comunidade ou não é possível a união porque o vértice não decidiu juntar-se a nossa comunidade. Mas não sei se esta seria a melhor porque os vértices poderiam passar vários passos a tentar unir as comunidades e uma comunidade maior atrasaria todas as outras.

\begin{figure}
\includegraphics[width=20mm]{esboco4}
\caption{Segundo grafo com comunidades\label{fig:esb4}}
\end{figure}

\newpage

Voltando ao exemplo do grafo quadrado, após a inicialização e o primeiro cálculo do ganho, ou seja, no superstep 2. Pelo o que vejo existem 4 casos: é

\begin{enumerate}
	\item Um vértice recebe a mensagem do vértice que o levou a mudar de comunidade e as suas comunidades são iguais (\underline{a} recebe uma mensagem de \underline{b} ou vice versa). Neste caso nada deveria ser feito.
	\item Um vértice recebe a mensagem do vértice que o levou a mudar de comunidade e as suas comunidades são diferentes (\underline{c} recebe uma mensagem de \underline{b} após este ter mudado para \underline{a}. Nesta caso o vértice que recebeu a mensagem deve mudar para a comunidade.
	\item Um vértice recebe uma mensagem de um vértice que pertence a sua comunidade mas não foi o vértice que o fez mudar (\underline{a} recebe uma mensagem de \underline{d} ou \underline{b} recebe uma mensagem de \underline{c}). Neste caso o vértice deve atualizar o $\sum_{tot}$ da sua comunidade.
	\item Um vértice recebe uma mensagem de um vértice que não pertence a sua comunidade diretamente (\underline{c} recebe uma mensagem de \underline{d} ou vice versa). Neste caso as duas vértices devem atualizar os respetivos $\sum_{tot}$ para incluir o outro.
\end{enumerate}


Sendo que a mensagem inclui o ID do vértice, o seu $K_i$, um histórico(ID comunidade, $\sum_{tot}$ da comunidade) das suas comunidades e também um histórico das comunidades para cada adjacente são levantadas algumas questões: 
\begin{itemize}
	\item Como é que a união poderia ser feita? 
	\item Tendo estes dados como seria possível distinguir o caso 1 do 3? 
	\item E seria possível atualizar os dados corretamente no caso 4?
	\item Se não for possível atualizar os dados corretamente neste passo os vértices devem ignorar as mensagens que vêm de vértices que não fazem parte da sua comunidade direta ou indiretamente e só começar a recalcular o ganho após? Se sim como seria possível detetar este caso?
\end{itemize}

A questão principal jaz em como poderia fazer a união entre as várias comunidades dos adjacentes tendo o seu histórico de comunidades? Não encontro nenhuma forma de o fazer.
\end{document}