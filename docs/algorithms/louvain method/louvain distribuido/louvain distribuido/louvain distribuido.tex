\documentclass[a4paper,10pt]{report}
\usepackage[utf8]{inputenc}
\usepackage{mathtools}
\usepackage{algorithm}
\usepackage{algorithmic}

\begin{document}
Cada vértice tem como valor:
\begin{itemize}
	\item \verb|id| o seu identificador único.
	\item \verb|deg| A soma do grau de todas as suas arestas. ($K_i$).
	\item \verb|tot| O total da soma dos graus das arestas incidentes aos vértices da sua comunidade. ($\sum_{tot}$).
	\item \verb|hub| O id do concentrador da sua comunidade.
\end{itemize}

O primeiro passo do algoritmo foi dividido em inicialização e 3 fases que se repetem até for atingido a modularidade máxima, sendo que uma iteração é uma passagem destas 3 fases.

\begin{algorithm}[H]
\begin{minipage}{\textwidth}
\caption{Louvain \textit{Method} Distribuido Passo 1}
\label{alg:lmdPasso1}
\begin{enumerate}
  \item Inicialização
			\begin{enumerate}
				\item O \verb|hub| de todos os vértices passa a ser o \verb|id| do próprio vértice e \verb|tot| 				recebe \verb|deg|
			\end{enumerate}
	\item Primeira Fase
	\begin{enumerate}
	\item Se não for a primeira vez que está na primeira fase o vértice deve substituir o seu \verb|tot| e \verb|hub| por aqueles recebidos na mensagem
	\item Vértice envia os seus adjacentes o seu \verb|id|, \verb|tot|, \verb|hub| e custo da aresta que o liga ao adjacente\footnote{Se as arestas não tiverem custo o custo considerado é 1}.
	\end{enumerate}
	
  \item Segunda Fase -  Ao receber uma mensagem o vértice deve:
  
		\begin{enumerate}
				\item Se não for a primeira vez que está na segunda fase o vértice deve verificar se não existiram mudanças de comunidade na segunda fase anterior e
						\subitem Se não for a primeira vez que está no primeiro passo e nunca ocorreram mudanças em nenhumas das segundas fases deste passo deve parar o algoritmo
						\subitem Caso contrário, deve começar o segundo passo descrito no algoritmo~\ref{alg:lmdPasso2}
				\item Calcular a melhor comunidade utilizando a equação ((Ref para eq))\footnote{Quando se calcula se a comunidade atual é a melhor comunidade é necessário remover o \textit{deg} atual do \textit{tot} da comunidade e, se o \textit{tot} passar a ser 0 então o ganho de modularidade será 0 de modo a impedir que os vértices nunca mudem de comunidade}. 
				Em caso de empate	escolher a comunidade em que \verb|hub| seja o menor.
				\item Em iteração pares o vértice não move para comunidades cujo \verb|hub| seja menor que o seu \verb|hub| atual.
				\item Registar globalmente se mudou de comunidade
				\item Atualizar \verb|hub| com o \verb|hub| da melhor comunidade e enviar para o \verb|hub| uma mensagem contendo \verb|id| e \verb|deg|.
		\end{enumerate}
		
  \item Terceira Fase - Os \textit{hub}, ao receberem as mensagens, devem:
\begin{enumerate}
		\item Calcular um novo \verb|tot| a partir dos \verb|deg| recebidos.
		\item Se receber uma mensagem não proveniente do próprio cujo \verb|id| seja igual ao seu \verb|hub| atual o \textit{hub} que tenha o maior \verb|id| adiciona ao \verb|tot| o seu \verb|deg| e regista que terá que enviar uma mensagem a si próprio e elege-se como novo \verb|hub|.
		\label{prt:cycle}
		\item Se não pertencer à comunidade que está a agregar no momento elege como novo representante desta comunidade o que tiver o menor \verb|id|.
		\label{prt:same}
		\item Enviar uma mensagem a todos os elementos da comunidade que agregou com o novo \verb|tot| da comunidade. Enviando a si mesmo caso seja necessário.
\end{enumerate}
\end{enumerate}

	\end{minipage}
\end{algorithm}

É de notar que, terceira fase, tanto no ponto~\ref{prt:cycle}, no caso do vértice receber uma mensagem do \verb|hub| que ele elegeu na fase anterior, e no ponto~\ref{prt:same} o vértice não recebe uma mensagem de si próprio sendo que este, em~\ref{prt:cycle}, tem que garantir que envia uma mensagem de atualização a si próprio e em~\ref{prt:same} um outro vértice irá enviar-lo a mensagem de atualização. Sendo que estes dois casos nunca ocorrem em simultâneo.

\paragraph{}
O segundo passo assenta sobre o primeiro, tendo apenas duas fases, que sobrepõem a segunda e terceira fase do primeiro passo.

\begin{algorithm}[H]
\caption{Louvain \textit{Method} Distribuido Passo 2}
\label{alg:lmdPasso2}
\begin{enumerate}
	\item Primeira Fase
	
	\begin{enumerate}
		\item O vértice cria uma estrutura de tuplos (\verb|hub|,\verb|deg_s|), onde para cada comunidade presente nas mensagens recebidas, \verb|deg_s| será a soma de todos os \verb|deg| dos elementos dessa comunidade conhecidos através destas mensagens.
		\item Os vértices removem as suas arestas e enviam esta estrutura ao concentrador.
	\end{enumerate}
	
	\item Segunda Fase
	
	
	\begin{enumerate}
		\item O concentrador, com uma estrutura similar à da primeira fase agrega os valores de todos os elementos da comunidade.
		\item Para cada elemento da estrutura o concentrador deve criar novas arestas em que o \verb|id| de destino é o \verb|hub| do elemento e o peso da aresta é \verb|deg_s|, atualizando o seu \verb|deg|. Envia a si próprio uma mensagem com um \verb|tot| igual ao \verb|deg| agora calculado. Nota que o peso total de arestas é alterado neste ponto e deve ser atualizado.
		\item Recomeçar o passo 1.
	\end{enumerate}
\end{enumerate}
\end{algorithm}

Esta versão distribuída tenta criar comunidades aproximadas às das criadas pela versão não distribuída mas não está garantido que as comunidades serão exatamente iguais. Isto é porque a resolução de ciclos pode levar a que vértices que pertenceriam a sua própria comunidade mudem para a comunidade de um adjacente e dado a natureza distribuída do algoritmo estarão a ser formadas várias comunidades concorrentemente o que significaria comunidades intermédias diferentes.

\paragraph{}
Existe uma versão distribuída deste algoritmo(ref:https://github.com/Sotera/distributed-louvain-modularity) que usa uma ideia parecida de dividir o algoritmo em 3 fase mas devido a diferença na resolução de ciclos as comunidades formadas serão diferentes das comunidades formadas pelo nosso algoritmo, sendo que as formadas pelo nosso aproximam-se mais daquelas formadas pelo algoritmo de Louvain original. O nosso algoritmo tem também um número menor de \textit{supersteps} no primeiro passo, sendo que o algoritmo em (ref:https://github.com/Sotera/distributed-louvain-modularity) usa o Map Reduce do Hadoop para realizar o segundo passo.

Possivelmente serão feitos testes comparando o tempo de computação dos dois algoritmos.
\end{document}