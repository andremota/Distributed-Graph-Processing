\documentclass[a4paper,10pt]{report}
\usepackage[utf8]{inputenc}
\usepackage{mathtools}
\usepackage{algorithm}
\usepackage{algorithmic}

\begin{document}
Cada vértice tem como valor:
\begin{itemize}
	\item \verb|id| o seu identificador único.
	\item \verb|w_deg| A soma do grau de todas as suas arestas. ($K_i$).
	\item \verb|tot| O total da soma dos graus das arestas incidentes aos vértices da sua comunidade. ($\sum_{tot}$).
	\item \verb|rep| O id do representante da sua comunidade.
\end{itemize}

O algoritmo é assim definido:

\begin{algorithm}
\caption{Louvain \textit{Method} Distribuido}
\begin{enumerate}
	\item O \verb|rep| de todos os vértices passa a ser o \verb|id| do próprio vértice
	\item Vértice envia os seus adjacentes o seu \verb|id|, \verb|tot| e \verb|rep|
\end{enumerate}
\end{algorithm}
\end{document}