\chapter{Apache Giraph e Apache Hama interface prográmavel para grafos.}

A interface de programação disponibilizada pelo Giraph e pelo Hama, apesar de ambas se basearem no modelo de programação \textit{Bulk Synchronous Parallel}, apresentam algumas diferenças. Tendo em conta que um dos objectivos deste trabalho é a criação de um módulo comum para a criação de algoritmos para as duas plataformas, procedeu-se então ao levantamento das suas interfaces de programação e à sua comparação. Os tipos comparados neste capítulo não são partes integrantes do modelo \textit{Bulk Synchronous Parallel} mas sim das plataformas comparadas.

\section{Master Compute}
\textit{MasterCompute} representa um vértice mestre que pode ser usada para controlar o algoritmo e as outras vértices. As plataformas que o implementam têm normalmente um método \texttt{compute()} sem parâmetros que corre entre \textit{supersteps} podendo ser utilizado para controlar a computação.


O Giraph apresenta a classe \texttt{MasterCompute} que permite implementar lógica que irá correr no \texttt{Master}\footnote{Master é o \textit{node} computacional que coordena a computação.} e apresenta o método \texttt{compute}. Pode ser usada como forma de controlar as fases de um algoritmo (mudando a \texttt{Computation} usada por exemplo) e consegue comunicar com os vértices usando os agregadores. Não existe uma classe equivalente no Hama.

\newpage
\section*{Input/Output}
No \textit{input} é onde é definido os vários leitores para os vários formatos que os ficheiros podem ter e onde são criados os vértices e as suas arestas. Os leitores, dependente do formato, podem também atribuir aos vértices um valor e às arestas um peso.
O \textit{output} é quem define	 o formato que terá o resultado final.
\subsection*{Apache Giraph}
%TODO - Mudar tipos do giraph!!!!!!
O Giraph baseia-se no modelo de I/O do Apache Hadoop estando construído por cima deste mesmo modelo.
A API que é disponibilizada define para os vértices e arestas os \texttt{InputFormat}, \texttt{InputReader}, \texttt{OutputFormat} e \texttt{OutputWriter}.
Os \texttt{InputFormat} têm a responsabilidade de gerar as divisões lógicas, que depois serão usadas para repartir os dados pelos vários
\textit{workers}\footnote{\textit{slave nodes}}. É no \texttt{InputFormat} que é criado o \texttt{VertexReader}, que irá depois ser usado para criar os vértices.
A criação do vértices pode envolver a utilização de um \texttt{RecordReader} do Hadoop onde serão lido os dados necessários a partir da informação dada pelos \texttt{InputSplit}.

O output do Giraph é controlado pelos vários \texttt{OutputFormat} que definem um método para criar um \texttt{OutputWriter}. O \texttt{OutputWriter} pode usar um \texttt{RecordWriter}
para escrever os vértices para um output.

\subsection*{Apache Hama}

O Hama também se baseia no modelo do Apache Hadoop mas constrói um modelo paralelo a este.
Este modelo consiste em \texttt{InputFormat}, \texttt{OutputFormat}, \texttt{VertexInputReader}, \texttt{VertexOutputReader}. Os \texttt{InputFormat} neste modelo criam \texttt{RecordReader} 
que vão ler os dados que vão ser posteriormente mapeados em pares chave-valor. O \texttt{VertexInputReader} irá receber o par chave-valor para criar os
vértices.

O output é gerado a partir de informação proveniente dos vértices utilizando o \texttt{BSPeer} para agregar pares chave/valor que iram
ser escritos pelo \texttt{VertexOutputWriter}. O formato do output é controlado pelo \texttt{OutputFormat} que é especificado na
configuração do \textit{job}.

%O output é gerado a partir da informação proveniente dos vértices utilizando os BSPeers do HAMA que internamente usam o OutputFormat 
%definido pelo utilizador.

\subsection*{Comparação}

Apesar de ambos o modelos de I/O serem semelhantes existem algumas diferenças devido ao facto do Giraph utilizar as implementações 
do Hadoop enquanto que o Hama não. 

Em relação ao input a principal diferença é que no \texttt{InputFormat} o Giraph cria um \texttt{Reader} para o \textit{input} desejado enquanto que 
o Hama cria um \texttt{RecordReader}. O Reader do Giraph cria o tipo desejado consoante o input passado (um par chave/valor). O \texttt{RecordReader}
do Hama irá mapear o input para um tuplo chave/valor que irá ser passado a um \texttt{InputReader} que foi previamente configurado no \textit{job}. O \texttt{InputReader}
no Hama tem a mesma funcionalidade de que o \texttt{InputReader} do Giraph.

Com o output acontece o mesmo que com o input em ambas as plataformas. No Giraph seleciona-se um \texttt{OutputFormat} para criar um \texttt{OutputWriter} de modo
a escrever o que é desejado. O Hama seleciona-se um \texttt{OutputFormat} que cria um \texttt{RecordWriter} de modo a obter um tuplo chave/valor que irá ser
passado a um \texttt{OutputWriter} para este gerar o output.

\newpage
\subsection{Combinadores}
O envio de mensagens entre vértices pode envolver o custo do envio sobre a rede ou pode existir um grande número de mensagens. Se o tipo de mensagem demonstrar uma propriedade comutativa é possível combinar várias mensagens numa única e enviar apenas esta mensagem, reduzindo o custo associado com o envio de mensagens. Os \textit{Combiners} servem para tal, combinando mensagens destinadas a um vértice.

\subsubsection*{Apache Giraph}
Esta plataforma apresenta uma interface de combinadores binários que define dois métodos \texttt{combine} e \texttt{createInitialMessage}.

O método \texttt{combine} é o método principal e recebe como parâmetro um identificador do vértice a que se destina as mensagens e duas mensagens a combinar.

O método \texttt{createInitialMessage} deve retornar uma mensagem cujo valor seja um elemento neutro da operação do combinador (ex: Um combinador para somas deve retornar 0).

Quando é especificado um combinador, as mensagens armazenadas são combinadas de modo a que o vértice receba apenas uma mensagem.

Adicionalmente, é possível mudar o tipo de combinador usado durante a computação no MasterCompute.

\paragraph{Exemplo}
\begin{verbatim}
public class MinIntCombiner extends MessageCombiner<LongWritable, IntWritable>{
	@Override
	public void combine(LongWritable vertex, IntWritable originalMessage, 
IntWritable otherMessage) {
		originalMessage.set(Math.min(originalMessage.get(), 
otherMessage.get()));
	}
	@Override
	public IntWritable createInitialMessage() {
		return new IntWritable(Integer.MAX_VALUE);
	}
}
\end{verbatim}

Com este combinador o vértice de destino apenas receberia a mensagem que 
tivesse o menor valor.

\subsubsection*{Apache Hama}
Esta plataforma apresenta uma interface de combinador de mensagens múltiplas que define o único método \texttt{combine}.

O método \texttt{combine} recebe uma sequência de mensagens destinadas a um 
vértice e deve retornar uma única mensagem como o resultado da sua operação de 
combinação.

\paragraph{Exemplo}
\begin{verbatim}
public class MinIntCombiner extends Combiner<IntWritable>{
	@Override
	public IntWritable combine(Iterable<IntWritable> messages) {
		IntWritable writable = new IntWritable(Integer.MAX_VALUE);
		for(IntWritable msg : messages){
			writable.set(Math.min(writable.get(), msg.get()));
		}	
		return writable;
	}
}
\end{verbatim}
Para este exemplo, um vértice apenas receberia a mensagem que tivesse menor 
valor.

\subsubsection*{Comparação}
As diferenças entre os combinadores Giraph e Hama resulta principalmente do tipo de combinação pedida ao programador. Sendo que no Giraph é apenas necessário combinar duas mensagens de cada vez mas no Hama é possível a combinação de várias mensagens em simultâneo.

Isto pode levar a código de combinadores mais simples no Giraph. No outro, lado a API do Giraph pede a implementação de dois métodos, em comparação com o Hama onde existe apenas um.

No Hama só é possível usar um tipo de combinador por computação mas no Giraph podem ser usados vários.

Para finalizar, o método \texttt{combine} do Giraph recebe também o \textit{id} do vértice, algo que pode ser contra-intuitivo visto que operações de combinação raramente dependem do vértice de destino.


\newpage

\subsection{Agregadores}
\label{ss:agreg}
  Por vezes os algoritmos aplicados sobre redes necessitam de comunicar um 
estado de um \textit{superstep} para o seguinte, daí a necessidade de existirem 
agregadores. Os agregadores são normalmente uma maneira de se obter um estado 
global, podendo esse estado ser um resultado computacionalmente calculado 
dependente dos valores agregados.

  %TODO - acabar intro?! 
  \subsubsection*{Apache Giraph}
    No Giraph os agregadores para puderem ser usados têm de ser registados no 
\texttt{MasterCompute}, usando o método \texttt{registerAggregator} e
    passando o nome do agregador e a classe do tipo que se quer registar. 
Pode-se aceder aos agregadores durante vários estágios da computação.
    Os valores dos agregadores podem ser obtidos em \texttt{MasterCompute}, 
\texttt{Vertex} e no \texttt{WorkerContext}\footnote{Permite 
especificar o que fazer em vários estágios da computação. Os estágios são: 
antes de um \textit{superstep}, depois de um \textit{superstep}, antes do 1º \textit{superstep} e depois do 
último \textit{superstep}.} usando o método \texttt{getAggregatedValue} e 
passando
    o nome do agregador.
    
    Durante os vários estágios do WorkerContext e em \texttt{Vertex.compute()} é possível agregar valores chamando o 
método \texttt{aggregate} passando o nome do agregador e o valor que irá ser agregado.
    
    No Giraph existe dois tipos de agregadores sendo um deles persistente e o 
outro regular. Nos agregadores regulares é chamado um método
    designado por \texttt{reset} que irá repor o valor inicial do agregador 
para cada \texttt{superstep}. Nos agregadores persistentes os seus valores vão 
durar
    para toda a aplicação. Esta distinção é feita no registo do agregador chamando \texttt{registerAggregator} 
    ou \texttt{registerPersistentAggregator}.
    
\paragraph{Exemplo}
\begin{verbatim}
public class GiraphDoubleSumAggregator extends BasicAggregator<DoubleWritable>{

	public void aggregate(DoubleWritable val) {
		getAggregatedValue().set(val.get()+getAggregatedValue().get());
	}

	public DoubleWritable createInitialValue() {
		return new DoubleWritable(0);
	}

}
\end{verbatim} 

Seria possível com este agregador manter o estado de um \textit{superstep} para 
o seguinte ou então manter o estado para toda a computação dependendo de como 
foi registado. Caso não seja persistente e tendo 5 vértices a agregar o valor 1 
no \textit{superstep} 1, no \textit{superstep} 2 seria possível observar o 
valor 5 para este agregador. No \textit{superstep} 2 se não houvesse nenhum 
vértice que agregasse um valor então o valor observado no \textit{superstep} 3 
seria 0. Caso fosse persistente, a única diferença é que no \textit{superstep} 
3 ainda era observável o valor 5.

\subsubsection*{Apache Hama}
    Para se usar agregador no Hama basta registar no \texttt{GraphJob} a classe, que irá agregar os valores, usando o método
    \texttt{setAggregatorClass}. Durante os vários \textit{supersteps} os 
vértices podem chamar o método \texttt{aggregate} passando um identificador 
único
    e o valor que irá ser agregado. Para aceder ao valor agregado de um 
agregador é usado o método \texttt{getAggregatedValue}, passando o seu 
    identificador único.
    
    O Hama permite implementar tipos de agregador, por exemplo, implementado 
a interface \texttt{Aggregator}.
    Esta classe é parametrizada com o tipo de valor que se quer agregar e 
contém um conjunto de métodos que podem ser redefinidos.
O método com mais relevância é o método \texttt{aggregate} que recebe um 
valor para agregar e um \textit{id} para indicar qual o agregador.

\paragraph{Exemplo}
\begin{verbatim}
 public class HamaDoubleSumAggregator implements Aggregator<DoubleWritable>{

	private DoubleWritable value = new DoubleWritable(0d);
	@Override
	public void aggregate(DoubleWritable val) {
		if(val!=null)
			value.set(value.get()+val.get());
		
	}

	@Override
	public DoubleWritable getValue() {
		return value;
	}
}
\end{verbatim}

  Com este agregador apenas é possível agregar globalmente um valor 
\textit{double} de um \textit{superstep} para o seguinte.

  \subsubsection*{Comparação}
  
  Apesar da ideia subjacente nas plataformas Hama e Giraph ser 
relativamente semelhante, existem algumas diferenças quanto à implementação. 
Ambas as plataformas
  permitem o registo de agregador e de os afetar em vários estágios da computação. 
  
  Para que os agregadores no Giraph tenham o mesmo comportamento que os presentes no Hama, os agregadores têm de ser regulares porque o Hama não tem agregadores persistentes. O Hama, ao contrário do Giraph, não permite o registo de agregadores parametrizados com um tipo diferente do tipo das mensagens.
  %O Hama simplifica a actualização do aggregator tendo uma interface mais completa do que o Giraph. Para que os aggregators no Giraph tenham o 
  %mesmo comportamento que os aggregators do Hama, em casos que é preciso valores pre-superstep e post-superstep, é necessário envolver 
  %entidades como o WorkerContext e fazer código para suportar as operações desejadas.
  


\subsection{Vértice}
\label{ss:vert}
Vértice é a classe que representa uma único vértice de um grafo e é a classe principal de um algoritmo pois é nesta que está o método \texttt{compute}, o método usado para definir o algoritmo. Através desta classe é possível aceder e alterar o estado do grafo e do próprio vértice. Um vértice pode votar para parar(\textit{halt}) o algoritmo e caso não receba uma mensagem nos próximos \textit{superstep} mantém-se \textit{halted}. Sendo que o algoritmo acaba quando todos os vértices estiverem \textit{halted}.
\subsubsection*{Apache Giraph}


O Giraph apresenta a interface \texttt{Vertex} mas para realizar computações deve ser deve ser feita a implementação da interface \texttt{Computation} para possibilitar a alteração da \texttt{Computation} usada durante a execução do algoritmo, sendo que os vértices serão sempre os mesmos.

A interface \texttt{Computation} tem cinco parâmetros genéricos sendo estes \texttt{I} - o tipo do identificador do vértice, \texttt{V} - o tipo do valor do vértice, \texttt{E} - o tipo do valor das arestas, \texttt{M1} - o tipo das mensagens de entrada e \texttt{M2} - o tipo das mensagens de saída.\texttt{Vertex} apresenta \texttt{I}, \texttt{V} e \texttt{E} equivalentes.

A interface \texttt{Computation} tem como método principal \texttt{compute}, o método que deve ser implementado para realizar a computação. Este método recebe como parâmetro o vértice atual e uma sequência de mensagens para este.

Através do método \texttt{compute} o programador pode aceder aos métodos do \texttt{Vertex} que lhe permite saber e alterar o estado do vértice e das suas arestas. Também lhe permite aceder aos outros métodos de \texttt{Computation} que lhe permite alterar o estado do grafo, adicionando ou removendo vértices e arestas e conhecer o estado geral do contexto. Por exemplo, é possível saber qual o \textit{superstep}, qual o número total de vértices existentes no \textit{superstep} atual ou informação relacionada com os valores agregados. É também possível enviar mensagens aos vértices vizinhos ou a um vértice em particular.

Em adição, a interface \texttt{Computation} apresenta os métodos \texttt{preComputation} e \texttt{postComputation} que permite escrever código que será executado antes de depois da computação, respetivamente.

As computações mais comuns devem derivar de \texttt{AbstractComputation} que já apresenta a implementação dos métodos utilitários de \texttt{Computation} ou da sua subclasse \texttt{BasicComputation} que apresenta apenas um tipo de mensagem.

É possível o uso de diferentes tipos de \texttt{Computation} diferentes durante a computação alterando o tipo no \texttt{MasterCompute}.

\paragraph{Exemplo}
\begin{verbatim}
public class SimpleComputation extends 
BasicComputation<LongWritable, IntWritable, NullWritable, IntWritable> {
  @Override
  public void compute(Vertex<LongWritable, IntWritable, NullWritable> vertex,
	Iterable<IntWritable> messages) throws IOException {
    if (getSuperstep() == 1) {
		      IntWritable writable = vertex.getValue();
			      for (IntWritable msg : messages) {
				        writable.set(Math.min(writable.get(), msg.get()));
      }
      vertex.voteToHalt();
			      return;
		  }
		  sendMessageToAllEdges(vertex, vertex.getValue());
	}}
\end{verbatim}

Com esta computação de um vértice, cada vértice no 1º \textit{superstep} 
enviará para os seus adjacentes o seu valor e no 2º \textit{superstep} irá 
escolher o menor valor entre os adjacentes e o próprio vértice.

\subsubsection*{Apache Hama}
A API de vértices do Hama é definida pela classe \texttt{Vertex} em que o método principal é \texttt{compute} que permite o mesmo que o método \texttt{compute} do Giraph. Recebe apenas uma sequência de mensagens destinadas ao vértice pois não necessita receber \texttt{Vertex} sendo o próprio um \texttt{Vertex}.

A classe \texttt{Vertex} tem três parâmetros genéricos sendo estes \texttt{V} - o tipo do identificador do vértice, \texttt{E} - o tipo do valor das arestas e \texttt{M} - o tipo do valor do vértice e das mensagens.


Adicionalmente é também necessário implementar o método \texttt{setup(Configuration conf)} onde é possível definir configurações extras.

\paragraph{Exemplo}
\begin{verbatim}
public class SimpleVertex extends Vertex<LongWritable, NullWritable, 
IntWritable>{
  public void compute(Iterable<IntWritable> messages) throws IOException {
	    if(getSuperstepCount() == 1){
        IntWritable writable = getValue();
        for(IntWritable msg : messages){
	         writable.set(Math.min(msg.get(), writable.get()));
        }
        voteToHalt();
        return;
      }
      sendMessageToNeighbors(getValue());
}}
\end{verbatim}

Este exemplo tem o mesmo comportamento que o algoritmo descrito para a 
computação em giraph em que cada vértice irá ficar com o mínimo valor na sua 
adjacência.

\subsubsection*{Comparação}


As duas \textit{frameworks} diferem essencialmente porque no Hama a computação é feita através da classe \texttt{Vertex} e o Giraph tem uma segunda classe específica para computação. Isto permite que no Giraph possam ser usado vários tipos diferentes de \texttt{Computation} e assim mais facilmente definir as várias fases de um algorítmo.
No Giraph é possível criar vértices em que o seu tipo de valor é diferente do tipo das mensagens, podendo o tipo de mensagens de entrada e saída ser diferentes, enquanto que o Hama assume que estes serão sempre iguais. Relativamente à criação de um vértice no Giraph necessita de uma chamada ao método \texttt{initialize}, enquanto que no Hama a inicialização é feita automaticamente.

\newpage
\section{Edge}

\subsection*{Apache Giraph}
O apache giraph para suportas arcos tem um conjunto de interfaces como Edge, MutableEdge,ReusableEdge. A interface de edge define as operações
elementares como obter o valor de um arco e o identificar do vértice a ela associado. A interface MutableEdge estende a interface Edge e define
um método para alterar o valor associado a um arco. ReusableEdge estende de MutableEdge e permite mudar o vértice que está associado ao arco.
A implementação dos arcos está feita nas classes DefaultEdge e EdgeNoValue (implementado ReusableEdge) e estas têm todos os métodos necessários ao uso de edges.

\subsection*{Apache Hama}
O Hama para suportar o conceito de arco tem uma classe Edge que suporta um conjunto de operações de obter o valor associado ao arco, mudar o seu valor,
obter o vértice vizinho e poder muda-lo. 

\subsection*{Comparação}
Apesar do Giraph ter um conjuto maior de tipos para suportar as diversas funcionalidades de Edge, ambas fazem essencialmente
o mesmo.

%\section{Resumo da Comparação da interface programável do Apache Giraph e Apache Hama}

Aggregators:
\begin{table}[H]
 \begin{tabular}{|| l | l | l | l |}\hline
  & \textit{Aggregator} regular & \textit{Aggregator} persistente & Tipos valor a agregar diferentes das mensagens\\\hline
  Hama & Sim & Não & Não \\\hline
  Giraph & Sim & Sim & Sim \\\hline
 \end{tabular}

 
 
\end{table}

