\newpage

\subsection{Agregadores}
  Por vezes os algoritmos aplicados sobre redes necessitam de comunicar um 
estado de um \textit{superstep} para o seguinte, daí a necessidade de existirem 
agregadores. Os agregadores são normalmente uma maneira de se obter um estado 
global, podendo esse estado ser um resultado computacionalmente calculado 
dependente dos valores agregados.

  %TODO - acabar intro?! 
  \subsubsection*{Apache Giraph}
    No Giraph os agregadores para puderem ser usados têm de ser registados no 
\texttt{MasterCompute}, usando o método \texttt{registerAggregator} e
    passando o nome do agregador e a classe do tipo que se quer registar. 
Pode-se aceder aos agregadores durante vários estágios da computação.
    Os valores dos agregadores podem ser obtidos em \texttt{MasterCompute}, 
\texttt{Vertex} e no \texttt{WorkerContext}\footnote{Permite 
especificar o que fazer em vários estágios da computação. Os estágios são: 
antes de um \textit{superstep}, depois de um \textit{superstep}, antes do 1º \textit{superstep} e depois do 
último \textit{superstep}.} usando o método \texttt{getAggregatedValue} e 
passando
    o nome do agregador.
    
    Durante os vários estágios do WorkerContext e em \texttt{Vertex.compute()} é possível agregar valores chamando o 
método \texttt{aggregate} passando o nome do agregador e o valor que irá ser agregado.
    
    No Giraph existe dois tipos de agregadores sendo um deles persistente e o 
outro regular. Nos agregadores regulares é chamado um método
    designado por \texttt{reset} que irá repor o valor inicial do agregador 
para cada \texttt{superstep}. Nos agregadores persistentes os seus valores vão 
durar
    para toda a aplicação. Esta distinção é feita no registo do agregador chamando \texttt{registerAggregator} 
    ou \texttt{registerPersistentAggregator}.
    
  \subsubsection*{Apache Hama}
    Para se usar agregador no Hama basta registar no \texttt{GraphJob} a classe, que irá agregar os valores, usando o método
    \texttt{setAggregatorClass}. Durante os vários \textit{supersteps} os 
vértices podem chamar o método \texttt{aggregate} passando um identificador 
único
    e o valor que irá ser agregado. Para aceder ao valor agregado de um 
agregador é usado o método \texttt{getAggregatedValue}, passando o seu 
    identificador único.
    
    O Hama permite implementar tipos de agregador, por exemplo, implementado 
a interface \texttt{Aggregator}.
    Esta classe é parametrizada com o tipo de valor que se quer agregar e 
contém um conjunto de métodos que podem ser redefinidos.
O método com mais relevância é o método \texttt{aggregate} que recebe um 
valor para agregar e um \textit{id} para indicar qual o agregador.

  \subsubsection*{Comparação}
  
  Apesar da ideia subjacente nas plataformas Hama e Giraph ser 
relativamente semelhante, existem algumas diferenças quanto à implementação. 
Ambas as plataformas
  permitem o registo de agregador e de os afetar em vários estágios da computação. 
  
  Para que os agregadores no Giraph tenham o mesmo comportamento que os presentes no Hama, os agregadores têm de ser regulares porque o Hama não tem agregadores persistentes. O Hama, ao contrário do Giraph, não permite o registo de agregadores parametrizados com um tipo diferente do tipo das mensagens.
  %O Hama simplifica a actualização do aggregator tendo uma interface mais completa do que o Giraph. Para que os aggregators no Giraph tenham o 
  %mesmo comportamento que os aggregators do Hama, em casos que é preciso valores pre-superstep e post-superstep, é necessário envolver 
  %entidades como o WorkerContext e fazer código para suportar as operações desejadas.
  
