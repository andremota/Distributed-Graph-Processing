\chapter{Modelo \textit{Bulk Synchronous Parallel}}

O \textit{Bulk Synchronous Parallel}(BSP) é um modelo de programação que 
surgiu para que se possa fazer processamento aproveitando-se recursos 
computacionais existentes utilizando técnicas de computação concorrente.

\section{Descrição do Modelo}

\begin{itemize} 
A base do modelo de programação BSP consiste no seguinte:
 \item Componentes de processamento.
 \item Comunicação entre as entidades envolvidas no processamento.
 \item Sincronização das entidades que estão a processar.
\end{itemize}

As componentes de processamento, no BSP, estão fortemente ligadas a 
processadores lógicos de um computador. Contudo, em plataformas que se baseiam 
no modelo BSP e que estão construídas de modo a funcionarem em ambientes 
distribuídos o termo componentes de processamento vai para além dos 
processadores lógicos. Normalmente, em plataformas distribuídas, entende-se por 
componentes de processamento os vários \textit{nodes} e para cada \textit{node} 
os seus respetivos processadores lógicos. Desta forma, as plataformas 
distribuídas podem se aproveitar ao máximo da paralelização em cada 
\textit{node} e da distribuição de trabalho por \textit{node}.

O modelo BSP


